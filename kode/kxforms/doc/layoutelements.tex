



\pagebreak 
\section{ KXForms Layout Elements Descriptions}
\label{layoutelements}




\subsection{ The \texttt{page} Element}
\label{page}
\begin{description}
 \item Description: Per default, all controls of a form are placed on one page. If there are too many of them the resulting GUI might be cluttered or might not fit on the screen. A possible solution is to spread the controls on several pages. This can be done using the \texttt{page} element. The choice of the multi-page widget is left to the application. 

A page number of -1 means that the control is put on a page after the last defined page.

 \item Possible Content: Integer \texttt{\{-1, 0, 1, ...\}}

 \item Default: \texttt{-1}

 \item Example: 

\begin{lstlisting}[caption=\texttt{page} Element]
<page>1</page>
\end{lstlisting}
\end{description}




\subsection{ The \texttt{position} Element}
\label{position}
\begin{description}
 \item Description: This element defines the position of the control on the page it belongs to. If the position is -1 it will just be appended at the bottom.

 \item Possible Content: Integer \texttt{\{-1, 0, 1, ...\}}

 \item Default: \texttt{-1}

 \item Example: 

\begin{lstlisting}[caption=\texttt{position} Element]
<position>1</position>
\end{lstlisting}
\end{description}




\subsection{ The \texttt{appearance} Element}
\label{appearance}
\begin{description}
 \item Description: This element tells the application how to render the control it is applied to. ``\texttt{full}'' indicates that all options should always be visible. ``\texttt{compact}'' means that an adequate number of options should be shown at once while ``\texttt{minimal}'' should result in a presentation that always shows just one option at once and thus takes up the minimum space.

 \item Possible Content: \texttt{full} | \texttt{compact} | \texttt{minimal}

 \item Default: \texttt{minimal}

 \item Example: 

\begin{lstlisting}[caption=\texttt{appearance} Attribute]
<appearance>Full</appearance>
\end{lstlisting}
\end{description}